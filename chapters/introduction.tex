\chapter{Introduction}
\label{chap:introduction}

Visualizing point clouds is an essential part of developing intelligent autonomous navigation systems like self-driving cars. It is used for debugging purposes alongside making intuitive user interfaces for developers and clients. Those interfaces are aimed at visualizing the terrain on which the vehicle is moving.

However, rendering raw data might significantly decrease the application performance. We propose two algorithms for accelerating point cloud rendering without noticeable visualization quality loss to reduce a high impact on computational resources.

This work describes the design and implementation of two methods that can process the point clouds and generate optimized data that can be visualized with higher performance than unprocessed data. This work also measures the performance of provided methods compares them.

The results of this study will contribute to point cloud and terrain scan processing methods. While most studies focus on visualizing and processing solid 3D models, this study aims to process raw point clouds retrieved from terrain scanners.

A point cloud often appears as a product of object or terrain scanning. Terrain point clouds usually contain a large amount of data (one gibibyte or more). Visualizing unprocessed terrain scans might have a heavy impact on performance, thus make the application unusable. This work aims at optimizing the point cloud data.

This paper is organized as follows: there is a literature review at the beginning. Next, in the Methodology chapter, we explain the methods on a conceptual level. After that, in the Implementation chapter, we describe the implementation details and peculiarities. In the Results and Discussion chapter, we provide the performance measurements and evaluation of the results.

This work is a part of the Innopolis Simulator project. Innopolis Simulator is software for simulating autonomous vehicles in different environments and conditions. It can simulate countryside terrains and cityscapes with buildings, pedestrians, vehicles, and other structures. Developers may use this simulator to debug and validate self-driving algorithms. It uses methods from this work to quickly create and integrate maps from terrain scans.

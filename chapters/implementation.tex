\chapter{Implementation}
\label{chap:implementation}

\graphicspath{{figs/implementation/}}

This chapter introduces the implementation details of the proposed methods. The methodology chapter describes methods on a conceptual level. First, this chapter presents the development platform and programming tools. Then it describes the peculiarities of methods implementation on the chosen platform.

\section{Development Environment}

The Innopolis Simulator project introduced several development and business constraints. The primary development constraint was the programming environment: developing the application using the Unity Platform, as the larger project already uses Unity.

\subsection{Unity Platform}

Unity Platform is a platform for creating and running 3D content \cite{UnityAbout}.  We use it for developing an application that visualizes the scanned locations on client devices.

There are two main reasons for using an existing graphics engine:

\begin{itemize}
    \item \textit{Platform-agnostic development.} Unity claims to support a wide range of devices and platforms, including desktop and mobile devices \cite{UnityMultiplatform}. This aspect is essential as the primary project targets mobile devices for on-the-road usage.
    
    \item \textit{Lower development costs.} Unity provides the functionality for visualization and scripting, which would require deep research or a team of highly skilled specialists to implement that functionality on our own.
\end{itemize}

Unity engine uses .NET runtime and C\# programming language for scripting API \cite{UnityManScripts}.

For implementing the point cloud visualization module, we use Unity 2020.2.


\section{Data Preprocessing}

Vehicle hardware records the lidar data in PCAP format. PCAP is an open network traffic capturing system that uses its own “.pcap” file format \cite{PcapSpec}. The lidar controller software receives the data in network packets and captures the incoming stream into a PCAP file. Lidar controller uses a specific VLP-16 PCAP structure that stores the lidar laser firing responses, firing time, and capturing coordinates \cite{Vlp16PcapSpec}.

\subsection{Converting to LAS format}

As mentioned above, PCAP file contains the recorded packet stream from the lidar at the time of capturing an area. It means that the data are stored as individual captures from various perspectives and at different times.

This step collects the data stream into the single point cloud, which represents the whole captured area. The resulting point cloud is stored in LAS format.

LAS is an open file format for storing 3D point cloud data developed by the American Society for Photogrammetry and Remote Sensing \cite{LasR15}. Unlike PCAP, the LAS format stores the points of the cloud as double floating-point numbers.

We use the VeloView software for converting the PCAP captured stream into LAS files.

\subsection{Importing LAS files into Unity}

As this project uses the Unity engine as a base, it is essential to have the ability to read LAS files within its programming environment. 

We use the LASZip.net library for reading LAS files in the C\# environment.

We created the scripted importer \cite{UnityManScriptedImporters} to instruct Unity about LAS file loading, thus, further importing routines can be done automatically by Unity both in the editor and run time.


\section{Method 1: Generating LODs}

The \autoref{chap:methodology} describes this method in detail in  \autoref{sec:generating_lods}. The method implementation includes point cloud loading, generating different levels of detail, and storing the generated LODs in the asset database for quick retrieval at run time.

\subsection{Storing different LODs in an asset database}

Unity engine supports storing and loading resources using the asset system \cite{UnityManAssetWorkflow}. Thus, it helps developers distribute application resources in a natively supported format by the Unity engine.

In Unity, each LOD should have its corresponding game object \cite{UnityManLod}. To systematically store generated point clouds with different LODs, we used the asset database API that helps to organize the stored resource library.

In a single point cloud object, the system creates a different asset for each chunk and each level of chunk detail. Thus, the total number of assets made is $N_{assets}=N_{chunks}\times N_{LODs}$.

The project utilizes the AssetDatabase API \cite{UnityManAssetDb}, provided by Unity. First, the algorithm creates a folder for storing the database. Next, as the chunk is generated with requested LODs, the corresponding assets are created.

\begin{lstlisting}[language={[Sharp]C}, caption=Asset database API usage example.]
AssetDatabase.CreateFolder(directorypath, name);
AssetDatabase.CreateAsset(data, filename);
\end{lstlisting}

\subsection{Applying LODs}

Unity requires each LOD to be a separate game object that corresponds to a parent game object. From the other perspective, the main game object has several supplementary game objects that correspond to different levels of detail. The LOD Group component manages these LOD game objects \cite{UnityManLodGroup}. Each chunk has its own LOD group that manages chunk detail levels based on the distance to the viewer.

Example LOD settings are provided in the \autoref{tab:impl:lod-settings}.

\begin{table}[ht]
    \centering
    \begin{tabular}{l|l}
    Distance (abstract engine units) & LOD (stride) \\ \hline
    \textless 100 & 1 \\
    100 to 200 & 8 \\
    200 to 300 & 24 \\
    \textgreater 300 & Not rendered
    \end{tabular}
    
    \caption{LOD distances and their corresponding levels.}
    \label{tab:impl:lod-settings}
\end{table}

The shown settings helped us efficiently utilize the computational resources and significantly improve the performance. The Results chapter shows the detailed performance comparison and results.


\section{Method 2: Creating a Mesh}

The \autoref{chap:methodology} describes this method in detail in \autoref{sec:creating_mesh}. The method implementation includes creating the Unity mesh structure that the engine will render at the run time.

\subsection{Initializing the vertex grid}

Vertex grid is a grid-aligned set of vertices that will later form the mesh. We initialized the vertex grid as follows:

\lstinputlisting[language={[Sharp]C}, caption=Vertex grid initialization.]{code/vertex-grid-init.cs}

Note that the second coordinate is Z, as Unity treats Y-axis as up-facing. We initialized the grid as a one-dimensional array as it will be easier to iterate over vertices later.

The next step is to fill the vertex grid and the occurrence map. It is done by iterating over the array of points.

\lstinputlisting[language={[Sharp]C}, caption=Filling vertex grid.]{code/fill-vertex-grid.cs}

The system creates a list of triangles. It will contain the vertex indices that will form triangles. Next, the algorithm iterates over the occurrence map and records the triangles into the list.

\lstinputlisting[language={[Sharp]C}, caption=Generating triangles.]{code/generate-triangles.cs}

After the triangle list has been filled, we create the mesh object. The Mesh type represents a mesh in the Unity engine \cite{UnityScriptingMesh}. It requires triangles and vertices. Optionally, some enhancements can be added, like texture, material, or shaders.

\lstinputlisting[language={[Sharp]C}, caption=Mesh creation example.]{code/create-mesh.cs}

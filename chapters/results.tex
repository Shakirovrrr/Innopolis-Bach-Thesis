\chapter{Results}
\label{chap:implementation}

\graphicspath{{figs/implementation/}}

This chapter provides the performance metrics and resulting visuals of the described methods. \autoref{chap:methodology} describes these methods on a conceptual level, and \autoref{chap:implementation} unveils the implementation details.

First, this chapter describes the measuring methodology and testing environment. Then, it shows the collected results for this application without point cloud rendering acceleration and with acceleration methods applied.


\section{Measuring methodology}

\subsection{Testing environment}

We collected the results from the application running on a computer with the following technical specifications:

\begin{itemize}
    \item CPU: AMD Ryzen 7 3750H @ 2.3GHz, 4 cores (8 threads)
    \item GPU: NVIDIA GeForce GTX 1650
    \item RAM: 16 GB DDR4
\end{itemize}

We used the Unity Editor 2020.2.2f1 to run the developed algorithms.

\subsection{Metrics}

To evaluate the result, we proposed the following metrics:

\begin{itemize}
    \item Framerate (frames per second, FPS) – this metric helps assess the algorithm's performance.
    \item Memory consumption (megabytes, MB) – this metric helps assess the algorithm's resource consumption.
    \item Visual quality (absence of defects, amount of detail saved) – this metric is subjective.
\end{itemize}

\subsection{Testing scenario}

It is crucial to provide the generalized testing scenario to make the testing unbiased. In terms of this work, we tested the single component of the larger software project. The \autoref{chap:introduction} gives a brief description of the major project.

\begin{enumerate}
    \item Select the algorithm and set the parameters.
    \item Import or re-import\footnote{The asset should be reimported in case it was imported earlier with different parameters.} LAS file with the test point cloud.
    \item Add the imported point cloud to the scene.
    \item Run the scene.
    \item Collect the metrics: the framerate in the Unity Editor, the memory usage in the operating system task manager or another resource monitor, and the rendered scene on the editor screen.
\end{enumerate}


\section{Collected results}

To minimize the environmental impact on application performance results, we made ten measurements for each testing scenario. We collected the following data by averaging ten measures during the testing.

\subsection{Generating LODs}

\subsubsection{Urban environment}

In this scenario, we measured the performance of the LOD generation algorithm on the urban scene that was captured from the part of Bangkok, Thailand.

\begin{table}[h]
    \centering
    \begin{tabular}{l|l|l|l}
    LODs & FPS & Memory (GB) & Visual quality \\ \hline
    800, 1600,   3200 & 25.1 & 3.2 & High \\
    400, 800,   1600 & 49.5 & 3.2 & Medium \\
    200, 800,   1600 & 65.3 & 3.2 & Low
    \end{tabular}
    
    \caption{Caption}
    \label{tab:results:}
\end{table}

%% TODO Pictures

\subsubsection{Landscape environment}

In this scenario, we measured the performance of the LOD generation algorithm in the countryside scene that was captured nearby the Innopolis, Tatarstan, Russian Federation.

%% TODO Table

%% TODO Pictures

\subsection{Mesh generation}

\subsubsection{Urban environment}

In this scenario, we measured the performance of the mesh creation algorithm on the urban scene that was captured from the part of Bangkok, Thailand.

%% TODO Table

%% TODO Pictures

\subsubsection{Landscape environment}

In this scenario, we measured the performance of the mesh generation algorithm in the countryside scene that was captured nearby the Innopolis, Tatarstan, Russian Federation.

%% TODO Table

%% TODO Pictures
